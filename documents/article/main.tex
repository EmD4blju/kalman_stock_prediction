\documentclass[12pt]{article}

%~ Packages
\usepackage{amsmath}   % For math
\usepackage{graphicx}  % For images
\usepackage{float}     % For stronger figure placement
\usepackage{hyperref}  % For links
\usepackage[T1]{fontenc}
\usepackage{geometry} % For margins
\usepackage[utf8]{inputenc} % Polish characters
\usepackage[polish]{babel} % Polish language support
\usepackage{times} % Use Times font
\usepackage{indentfirst} % Indent first paragraph after section
\usepackage{titlesec} % Customize section titles

%~ Document settings
\geometry{a4paper, margin=2.5cm}
% \renewcommand{\familydefault}{\sfdefault} % Switched to Times

% Adjust justification to prevent overfull boxes without hyphenation
\tolerance=1000
\emergencystretch=\maxdimen
\hyphenpenalty=10000
\hbadness=10000

%~ Section formatting
\titleformat{\section}{\Large\bfseries}{\thesection.}{1em}{}
\titleformat{\subsection}{\large\bfseries}{\thesubsection.}{1em}{}


%~ Document
\begin{document}

%~ Main Page
\begin{titlepage}
    \centering
    
    \includegraphics[width=0.4\textwidth]{/home/emdablju/Documents/projects/kalman_stock_prediction/documents/article/assets/PJATK_PL_sygnet.png}

    {\Large Polsko-Japońska Akademia Technik Komputerowych}
    
    \vspace{0.5cm}
    
    {\large Wydział Informatyki}
    
    \vspace{2cm}
    
    {\huge\bfseries Zastosowanie filtrów Kalmana do poprawy predykcji cen giełdowych przy użyciu sieci LSTM}
    
    \vspace{2cm}
    
    {\Large Praca Dyplomowa}
    
    \vspace{2cm}
    
    \begin{flushleft}
    \begin{tabular}{ll}
        \textbf{Autor:} & Mikołaj Warda (s28034) \\
        \textbf{Kierunek studiów:} & Informatyka \\
        \textbf{Specjalizacja:} & Data Science \\
        \textbf{Promotor:} & dr Sinh Hoa Nguyen Thi \\
    \end{tabular}
    \end{flushleft}
    
    \vfill
    
    {\large \today}
    
\end{titlepage}

%~ Table of Contents
\tableofcontents
\clearpage

%~ Abstract
\begin{abstract}
    
\end{abstract}
\clearpage

%~ Introduction
\section{Wstęp}

Prognozowanie cen akcji odgrywa kluczową rolę w finansach, wspierając inwestorów w podejmowaniu świadomych decyzji zarządzania swoim portfolio.
Chaotyczny charakter rynków sprawia jednak, że trafne przewidywanie notowań pozostaje trudnym zadaniem.
Tradycyjna analiza techniczna bywa niewystarczająca - jest wrażliwa na szum, a wnioski często zawierają element subiektywności.

W ostatnich latach dynamiczny rozwój uczenia maszynowego, zwłaszcza sieci neuronowych, 
znacząco zmienił podejście do modelowania danych czasowych (w tym giełdowych).
Architektury takie jak LSTM (Long Short-Term Memory) potrafią uchwycić złożone zależności i długookresowe relacje w danych, 
co czyni je obiecującymi narzędziami do prognozowania cen, jednak ich skuteczność nadal zależy od jakości danych wejściowych.
Głównym problemem, jest wysoka wrażliwość na losowe wahania, które są nieodłącznym elementem danych finansowych. 
Te zniekształcenia wynikają z nieprzewidywalnych zdarzeń rynkowych.
Model, zamiast uczyć się rzeczywistych trendów, modeluje wspomniane zakłócenia, 
obniżając swoją zdolność do generalizacji na nowych danych. 
W rezultacie, predykcje mogą być obarczone znacznym błędem, 
co podważa ich użyteczność w podejmowaniu decyzji inwestycyjnych.

Można temu zapobiegać stosując techniki filtracji na danych wejściowych.
Jednym z klasycznych narzędzi tego typu jest filtr Kalmana,
który może pełnić rolę modułu wygładzania i korekty obserwacji, podnosząc stabilność i dokładność predykcji.
Trenowanie sieci neuronowej na przefiltrowanych danych przy użyciu filtra Kalmana może zredukować wpływ szumu, 
pozwalając na lepsze uchwycenie istotnych wzorców i trendów w danych.

W ramach niniejszej pracy dyplomowej, podjęto się zbadania skuteczności zastosowania filtru Kalmana jako etapu przetwarzania danych dla modeli LSTM w kontekście prognozowania cen akcji spółki Amazon.com Inc (AMZN) .
Wybór tematu pracy wyniknął z chęci zdobycia wiedzy na temat działania architektury LSTM oraz zbadania wpływu filtracji danych na jakość prognoz.
Dodatkową motywacją była możliwość wszechstronnego rozwoju ze względu na złożony charakter problemu, łączącego zagadnienia z dziedziny uczenia maszynowego, analizy szeregów czasowych oraz teorii filtrów.

W dalszej części pracy, w sekcji \textit{"Podstawy teoretyczne"}, przedstawiono wszelkie niezbędne pojęcia związane z tematem pracy i przebiegiem badań ze szczególnym naciskiem na architekturę sieci LSTM oraz filtr Kalmana.
Następnie, w sekcji \textit{"Metodyka badań"}, opisano podejście badawcze, w tym przygotowanie danych, implementację modelu, metryki oceny oraz wykorzystane narzędzia.
Kolejna sekcja \textit{"Wyniki i dyskusja"} prezentuje uzyskane wyniki eksperymentów wraz z ich analizą i interpretacją.
Na zakończenie, w sekcji \textit{"Podsumowanie i wnioski"}, podsumowano przeprowadzone badania, przedstawiono kluczowe wnioski oraz zasugerowano kierunki dalszych badań w tym obszarze.

%~ Fundamentals
\clearpage
\section{Podstawy teoretyczne}
\subsection{Przewidywanie szeregów czasowych na rynkach finansowych}
Dane finansowe, takie jak np. ceny akcji, zawierają obserwacje tworzące szeregi czasowe.
Innymi słowy, są to dane, które reprezentują zmiany wartości w określonych odstępach czasu.
Formalnie szeregiem czasowym określa się zbiór uporządkowanych obserwacji w czasie, gdzie każda obserwacja jest powiązana z określoną chwilą czasową:
\begin{equation}
    X = \{x_1, x_2, x_3, \ldots, x_n\}
\end{equation}
gdzie \( x_i \) reprezentuje wartość obserwacji w czasie \( t_i \), a \( n \) to liczba obserwacji w szeregu czasowym.

Abstrahując od rynków finansowych, przedstawiono kilka przykładów szeregów czasowych z innych dziedzin, celem lepszego zobrazowania tej koncepcji:
\begin{itemize}
    \item \textbf{Energetyka:} Godzinowe zużycie energii elektrycznej
    \item \textbf{Meteorologia:} Codzienny pomiar temperatury
    \item \textbf{Sport:} Wyniki meczów drużyny na przestrzeni sezonu
\end{itemize}

Najczęściej spotykanym formatem danych finansowych są tzw. dane OHLC.
Przedstawione w takiej postaci informacje tworzą szereg czasowy, w którym każda obserwacja składa się z czterech części:
\begin{itemize}
    \item \textbf{Open (O):} Cena otwarcia - cena, po której dany instrument finansowy rozpoczął notowania w danym okresie.
    \item \textbf{High (H):} Cena najwyższa - najwyższa cena osiągnięta przez instrument finansowy w danym okresie.
    \item \textbf{Low (L):} Cena najniższa - najniższa cena osiągnięta przez instrument finansowy w danym okresie.
    \item \textbf{Close (C):} Cena zamknięcia - cena, po której instrument finansowy zakończył notowania w danym okresie.
\end{itemize}
Na potrzeby tej pracy, wykorzystano składnik \textit{Close}, ze względu na jego powszechne wykorzystanie w analizie i prognozie cen akcji.

Rysunek 1 przedstawia przykładowy wykres cenowy względem składnika \textit{Close} spółki Amazon.com Inc (AMZN) na przestrzeni kilku lat w odstępach dziennych. 
Dane przedstawione na wykresie są nieregularne i zawierają liczne fluktuacje, co jest charakterystyczne dla danych finansowych.
Ta wysoka zmienność stanowi główne wyzwanie dla modeli predykcyjnych, co motywuje do poszukiwania skutecznych metod filtrujących.
\begin{figure}[H]
    \centering
    \includegraphics[width=0.8\textwidth]{/home/emdablju/Documents/projects/kalman_stock_prediction/documents/plots/1_AMZN_stock.png}
    \caption{Wykres cenowy względem składnika \textit{Close} Amazon.com Inc (AMZN) w latach 2022-2025}
    \label{fig:amzn_stock_prices}
\end{figure}

Tradycyjnie analiza danych rynkowych opierała się o wskaźniki techniczne. 
Można je określić mianem narzędzi statystycznych reprezentujących różne aspekty zachowań cen.
W niniejszej pracy wykorzystano trzy wskaźniki techniczne, które opisano w następnych podsekcjach.

\subsubsection{Wskaźnik siły względnej (Relative Strength Index, RSI)}
Wskaźnik siły względnej (RSI) to popularny wskaźnik techniczny używany do oceny siły i prędkości zmian cen aktywów finansowych.
Wyraża się go wzorem:
\begin{equation}
    RSI = 100 - \frac{100}{1 + RS}
\end{equation}
gdzie \( RS \) (Relative Strength) to stosunek średnich wzrostów do średnich spadków cen w określonym czasie. Według badań, optymalny okres do obliczania RSI wynosi 14 dni.

Produktem końcowym RSI jest liczba z zakresu od 0 do 100, która interpretowana jest następująco:
\begin{itemize}
    \item Wartości powyżej 70 sugerują, że akcja jest wykupiona i może nastąpić spadek cen.
    \item Wartości poniżej 30 sugerują, że akcja jest wyprzedana i może nastąpić wzrost cen.
    \item Wartości pomiędzy 30 a 70 wskazują na neutralny stan rynku.
\end{itemize}

\subsubsection{Wstęgi Bollingera (Bollinger Bands)}
Wstęgi Bollingera to narzędzie analizy technicznej, dostarczające informacji o zmienności rynku. 
Składaja się z trzech linii na wykresie cenowym:
\begin{itemize}
    \item \textbf{Środkowa linia:} Prosta średnia krocząca (SMA) obliczona na podstawie cen zamknięcia w określonym czasie. Najczęsciej używanym okresem jest 20 dni.
    \item \textbf{Górna wstęga:} Obliczana jako suma wartości środkowej linii i dwukrotności odchylenia standardowego cen w tym okresie.
    \item \textbf{Dolna wstęga:} Obliczana jako różnica wartości środkowej linii i dwukrotności odchylenia standardowego cen w tym okresie.
\end{itemize}
Na podstawie ww. składowych można zinterpretować dwie nowe miary, które zostały wykorzystane w niniejszej pracy:
\begin{itemize}
    \item \textbf{Bollinger Bandwidth (BBW):} Miara szerokości wstęg Bollingera, obliczana jako stosunek różnicy między górną a dolną wstęgą do środkowej linii:
    \begin{equation}
        BBW = \frac{Upper Band - Lower Band}{Middle Line}
    \end{equation}
    \item \textbf{Bollinger \%B (BB\%):} Miara położenia ceny względem wstęg Bollingera, obliczana jako stosunek różnicy między ceną zamknięcia a dolną wstęgą do różnicy między górną a dolną wstęgą:
    \begin{equation}
        BB\% = \frac{Close - Lower Band}{Upper Band - Lower Band}
    \end{equation}
\end{itemize}

Opisane powyżej wskaźniki techniczne - RSI, BBW oraz BB\% - dostarczają cennych informacji o dynamice rynku. 
W tradycyjnej analizie mogą posłużyć do subiektywnej interpretacji przez analityków.
W nowoczesnych podejściach, opartych na modelach uczenia maszynowego, 
można je wykorzystać do wzbogacenia danych o dodatkowe cechy, co potencjalnie może poprawić jakość prognoz.

\subsection{Sieci neuronowe w prognozowaniu rynków finansowych}

Wraz z rozwojem mocy obliczeniowej i technik uczenia maszynowego, pojawiły się bardziej zaawansowane metody analizy i prognozowania w dziedzinie finansów.
Do najpopularniejszych podejść należą sieci rekurencyjne (RNN), w tym ich zaawansowane warianty, takie jak \textit{LSTM} (Long Short-Term Memory).
Sieci te są zdolne do uchwycenia złożonych wzorców i zależności w szeregach czasowych.

\subsubsection{Podstawy działania sieci neuronowych}
Ważnym kamieniem milowym w dziedzinie sieci neuronowych było wprowadzenie matematycznego modelu neuronu przez McCullocha i Pitts'a w 1943 roku.
Współcześnie można mówić o różnych architekturach takiego neuronu, jednak ich podstawowa zasada działania pozostaje podobna.

Neuron otrzymuje na wejściu sygnały \( x_1, x_2, \ldots, x_n \), które są ważone przez odpowiednie wagi \( w_1, w_2, \ldots, w_n \).
Następnie, sumuje te ważone sygnały i dodaje do nich wartość biasu \( b \):
\begin{equation}
    z = \sum_{i=1}^{n} w_i x_i + b
\end{equation}

Otrzymana wartość \( z \), zwana logitem (surowym wyjściem) neuronu, jest następnie przekształcana przez funkcję aktywacji \( f(z) \).
\begin{equation}
    y = f(z) = f(\sum_{i=1}^{n} w_i x_i + b)
\end{equation}

Funkcja aktywacji modyfikuje logit \( z \), produkując ostateczne wyjście neuronu \( y \).
Takie wyjście jest interpretowane jako poziom aktywacji neuronu.
W zależności od zastosowanej funkcji aktywacji, wyjście może przyjmować różne formy.
Tą różnice najprościej zilustrować na przykładzie dwóch popularnych funkcji aktywacji: skokowej (step function) oraz sigmoidalnej (sigmoid function).
\begin{itemize}
    \item \textbf{Funkcja skokowa:}
    \begin{equation}
        f(z) = 
        \begin{cases}
            1 & \text{jeśli } z \geq 0 \\
            0 & \text{jeśli } z < 0
        \end{cases}
    \end{equation}
    W tym przypadku, wyjście neuronu jest binarne. 
    Oznacza to, że neuron zostanie albo aktywowany (1), albo nieaktywowany (0), w zależności od tego, czy logit \( z \) przekracza pewien próg (w tym przypadku 0).
    \begin{figure}[H]
        \centering
        \includegraphics[width=0.6\textwidth]{/home/emdablju/Documents/projects/kalman_stock_prediction/documents/plots/step_figure.png}
        \caption{Wykres funkcji aktywacji skokowej}
        \label{fig:step_function}
    \end{figure}
    \item \textbf{Funkcja sigmoidalna:}
    \begin{equation}
        f(z) = \frac{1}{1 + e^{-z}}
    \end{equation}
    W tym przypadku, wyjście neuronu jest ciągłe i mieści się w zakresie od 0 do 1.
    Oznacza to, że neuron może przyjmować różne poziomy aktywacji. Czyni to sieć bardziej elastyczną i zdolną do modelowania złożonych zależności.
    \begin{figure}[H]
        \centering
        \includegraphics[width=0.6\textwidth]{/home/emdablju/Documents/projects/kalman_stock_prediction/documents/plots/sigmoid_figure.png}
        \caption{Wykres funkcji aktywacji sigmoidalnej}
        \label{fig:sigmoid_function}
    \end{figure}
\end{itemize}

Znając rolę funkcji aktywacji, można lepiej wyjaśnić działanie parametrów uczonych neuronu, czyli wag \( w_i \) oraz biasu \( b \).
Wagi określają, jak duży wpływ ma każdy sygnał wejściowy \( x_i \) na logit \( z \).
Poprzez modelowanie wag, w trakcie procesu uczenia, sieć może nauczyć się, które cechy wejściowe są bardziej istotne dla danego zadania.
Działanie wag na wyjście modelu można zilustrować na przykładzie funkcji sigmoidalnej. 

Rysunek \ref{fig:sigmoid_weights_combined} przedstawia wpływ różnych wartości wagi \( w \) na kształt funkcji sigmoidalnej w neuronie z jednym wejściem i bez biasu.
Po lewej stronie znajduje się wykres dla małej wagi \( w=0.1 \), gdzie funkcja zmienia się powoli i ma łagodne nachylenie.
Po prawej stronie znajduje się wykres dla dużej wagi \( w=2.0 \), gdzie funkcja zmienia się gwałtownie i ma strome nachylenie.
Ciekawym zjawiskiem jest to, że wraz ze wzrostem wartości wagi, funkcja sigmoidalna zaczyna przypominać funkcję skokową.

\begin{figure}[H]
    \centering
    \begin{minipage}{0.48\textwidth}
        \centering
        \includegraphics[width=\linewidth]{/home/emdablju/Documents/projects/kalman_stock_prediction/documents/plots/sigmoid_w_min_figure.png}
        % Można dodać pod-podpis, jeśli to konieczne, używając pakietu subcaption
    \end{minipage}\hfill
    \begin{minipage}{0.48\textwidth}
        \centering
        \includegraphics[width=\linewidth]{/home/emdablju/Documents/projects/kalman_stock_prediction/documents/plots/sigmoid_w_max_figure.png}
    \end{minipage}
    \caption{Wpływ wartości wagi (w) na kształt funkcji sigmoidalnej: (po lewej) \(w=0.1\), (po prawej) \(w=2.0\).}
    \label{fig:sigmoid_weights_combined}
\end{figure}

% %~ Research approach
% \section{Metodyka badań}

% %~ Results
% \section{Wyniki i dyskusja}

% %~ Summary
% \section{Podsumowanie i wnioski}

% %~ Bibliography
% \section{Bibliografia}

% %~ List of figures and tables
% \section{Spis rysunków i tabel}

% %~ Attachments
% \section{Załączniki}


\end{document}
