\documentclass[12pt]{article}

%~ Packages
\usepackage{amsmath}   % For math
\usepackage{graphicx}  % For images
\usepackage{hyperref}  % For links
\usepackage[T1]{fontenc}
\usepackage{geometry} % For margins
\usepackage[utf8]{inputenc} % Polish characters

%~ Document settings
\geometry{a4paper, margin=2.5cm}
\renewcommand{\familydefault}{\sfdefault}

%~ Document
\begin{document}

%~ Main Page
\begin{titlepage}
    \centering
    
    \includegraphics[width=0.4\textwidth]{/home/emdablju/Documents/projects/kalman_stock_prediction/documents/article/assets/PJATK_PL_sygnet.png}

    {\Large Polsko-Japońska Akademia Technik Komputerowych}
    
    \vspace{0.5cm}
    
    {\large Wydział Informatyki}
    
    \vspace{2cm}
    
    {\huge\bfseries Zastosowanie filtrów Kalmana do poprawy predykcji cen giełdowych przy użyciu sieci LSTM}
    
    \vspace{2cm}
    
    {\Large Praca Dyplomowa}
    
    \vspace{2cm}
    
    \begin{flushleft}
    \begin{tabular}{ll}
        \textbf{Autor:} & Mikołaj Warda (s28034) \\
        \textbf{Kierunek studiów:} & Informatyka \\
        \textbf{Specjalizacja:} & Data Science \\
        \textbf{Promotor:} & dr inż. Sinh Hoa Nguyen \\
    \end{tabular}
    \end{flushleft}
    
    \vfill
    
    {\large \today}
    
\end{titlepage}
\pagebreak

%~ Abstract
\section{Streszczenie}

%~ Introduction
\section{Wstęp}

%~ Fundamentals
\section{Podstawy teoretyczne i przegląd literatury}

%~ Research approach
\section{Metodyka badań}

%~ Results
\section{Wyniki i dyskusja}

%~ Summary
\section{Podsumowanie i wnioski}

%~ Bibliography
\section{Bibliografia}

%~ List of figures and tables
\section{Spis rysunków i tabel}

%~ Attachments
\section{Załączniki}


\end{document}
